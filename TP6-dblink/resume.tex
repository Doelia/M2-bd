\subsection{Préambule}\label{pruxe9ambule}

\subsubsection{Approche déscendante
(top-dowm)}\label{approche-duxe9scendante-top-dowm}

À partir d'un schéma global pour produire des schémas locaux -
Fragmentation horizontale - Fragmentation verticale - Fragmentation
hybride

\subsubsection{Approche ascendante
(down-top)}\label{approche-ascendante-down-top}

À partir des schémas locaux pour produire un schéma global

\subsection{Plusieurs types de liens :}\label{plusieurs-types-de-liens}

Lien privé portant sur un schéma utilisateur :

\begin{verbatim}
create database link nomDuLien connect to nomSchema identifie by motPasseSchema using 'nomService';
\end{verbatim}

Lien public défini globalement sur la base de donnés :

\begin{verbatim}
create public database link nomDuLien using 'nomService';
\end{verbatim}

Lien privé portant sur le schéma utilisateur interne user1 et la BD
master :

\begin{verbatim}
create database link monLien connect to user1 identified by user1 using 'master';
\end{verbatim}

Lien privé portant sur le schéma utilisateur interne user1 et la BD
licence :

\begin{verbatim}
create database link me_to_user1_licence connect to user1 identified by user1 using 'licence';
\end{verbatim}

\subsection{TP}\label{tp}

Etapes : 1. démarche d'intégration de schémas de bases de données
existants 2. démarche de décomposition

Schéma 1 : pcavalet / MASTER Schéma 2 : swouters / LICENCE Master global
: swouters / MASTER

\subsubsection{~Création les liens}\label{cruxe9ation-les-liens}

\begin{verbatim}
create database link sc1 connect to pcavalet identified by pikynau using 'MASTER';
create database link sc2 connect to swouters identified by wugaxu2 using 'LICENCE';
\end{verbatim}

Vérifier que ça marche :

\begin{verbatim}
select table_name from user_tables@sc1;
\end{verbatim}

\subsubsection{Requetes}\label{requetes}

les informations générales sur les virus

\begin{verbatim}
select * from virus@sc1;
\end{verbatim}

\subsubsection{Créer des vues}\label{cruxe9er-des-vues}

Vue dynamique : CREATE VIEW virus AS select * from virus@sc1;

Vue matérialisée : CREATE MATERIALIZED VIEW virus AS select * from
virus@sc1;
